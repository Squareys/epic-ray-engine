\documentclass[11pt,a4paper,notitlepage]{article}
\usepackage[utf8]{inputenc}
\usepackage{amsmath}
\usepackage{amsfonts}
\usepackage{amssymb}
\usepackage{graphicx}
\usepackage[left=2cm,right=2cm,top=2cm,bottom=2cm]{geometry}
\usepackage{listings}

\title{EpicRay - Design Document}
\author{Jonathan Hale}

\lstset{language=Java}

\newcommand{\code}{\texttt}

\begin{document}
\maketitle

%%%%%%%%%%%%%%%%%%%%%%%%%%%%%
% Introduction				%
%%%%%%%%%%%%%%%%%%%%%%%%%%%%%
\section{Introduction}
EpicRay is a modern Raycasting engine in Java. I tried to desing it for great versatility. 
EpicRay should offer most things as Interfaces/abstract classes and always one Implementation. See Renderer for example.

\section{Definitions}
\textbf{Representation} everything concerning the graphics and audio of the game.  \newline
\textbf{Logic} what works in the background, invisible to the user. Stuff like entities, world and physics. \newline
\textbf{Mechanics} special functionalities which can define a game.

%%%%%%%%%%%%%%%%%%%%%%%%%%%%%
% Resource Managment		%
%%%%%%%%%%%%%%%%%%%%%%%%%%%%%

\section{Resource Management}
How resources needed for representation are handled in EpicRay. Resources are the following.

\subsection{Resources}

\subsubsection{Bitmap}
Bitmap stores pixel data in an array. 

\subsubsection{Texture}
A Texture is a area on a bitmap which can be used to draw on walls, ceilings, floors, or sprites.

\subsubsection{Sound}
Sound is a sample that can be played.

\subsection{Resource Manager}
Resource Manager handles loading and unloading of the resources. It also containes lists of all the existing resources and can be optained through the Resource Manager at any time.

%%%%%%%%%%%%%%%%%%%%%%%%%%%%%
% Game Logic				%
%%%%%%%%%%%%%%%%%%%%%%%%%%%%%
\section{Game Logic}

\subsection{Entity}
Represents a \textit{thing} in the game like moving player, enemy or static destroyable barrel, that actually have a function. Purely eastetic things should be included in the game as sprites. As soon as it supports collision it isn't purely eastetic anymore, though.

\subsubsection{Player}
Just another entity controlled by the Game Handler.

\subsection{Tile}
A Block in the tile map. It containes a list of references to the entities curently in the tile and a list of decorating sprites. Also has a wall, ceiling and floor texture. Tiles can be opaque (blocks a ray completely, for transparent, or half transparent, opaque should be off) and solid (meaning entities collide with it).

\subsection{Tile Map}
An array of Tiles. Has getter and setter funktions for the tiles.

\subsection{World}
Contains everything that makes up the game-world. This means: entities, tilemap, background (?), and so on. No GUI or HUD.

\subsection{Input Handler}
Handles input such as mouse, keyboard and joystick/gamepad. Input Handler also acts as an abstraction layer for the input (''Joypad``).

\subsection{Game Handler}
Handles input-world-interaction and the HUD/GUI. The class doing this is also responsible for rendering and updating the world.

%%%%%%%%%%%%%%%%%%%%%%%%%%%%%
% Rendering 				%
%%%%%%%%%%%%%%%%%%%%%%%%%%%%%
\section{Rendering}
Renderers are defined through an abstract class. One implementation is given in EpicRay.

\subsection{Technical}
Since EpicRay uses raycasting for graphics, rendering is pretty strictly defined already. 

Following rendering order must be kept:
\begin{enumerate}
\item draw sky/background. 
\item draw world (walls/floor/ceiling)
\item draw sprites
\item apply filters.
\end{enumerate}

I was thinking about creating a depth buffer while rendering world and sprites. This would allow depth-dependant filters like DOF and would make Occlusion of sprites and walls alot easier.

Another thought was to create something like a Ray class for the EpicRayRenderer which stores a vertical stripe of pixels and calculates itself and then gets drawn to the rendering destiny bitmap.

\subsection{Raycasting}
Add a brief explanation of Raycasting here...

\subsection{Renderer}
The Renderer has a funktion to render a GameHandler object. 

 
\end{document}
